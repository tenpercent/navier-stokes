\documentclass[12pt]{article}
\usepackage[utf8]{inputenc}
\usepackage[russian]{babel}
\usepackage{amsmath}

\title{Вычислительный практикум: Численное моделирование нестационарного течения вязкого газа
с использованием неявной разностной схемы}

\begin{document}
\maketitle
\section{Задание 1}
\subsection{Постановка задачи}
Рассмотрим систему уравнений, описывающую одномерное нестационарное движение вязкого баротропного газа:

$
\begin{cases}
	\frac{\partial\rho}{\partial t} + \frac{\partial\rho u}{\partial t} = 0{,}\\
	\rho\frac{\partial u}{\partial t} + \rho u \frac{\partial u}{\partial x} + \frac{dp}{dx} = \mu \frac{\partial^2 u}{\partial x^2}{,}\\
	\rho(x{,} \:0) = \rho_0 (x),\\
	u(x{,} \:0) = u_0 (x),\\
	u(0, \:t) = u (X, \:t) = 0.
\end{cases}
$

$\Omega =\left[0{,} \:X\right] \times\left[0{,} \:T\right]$ ~--- область определения исходных функций,

$\mu$ ~--- вязкость газа, положительная константа,

$\rho (x{,} \:t)$ ~--- плотность газа, $(x{,} \:t) \in \Omega$,

 $u (x{,}\: t)$ ~--- скорость газа, $(x{,}\: t) \in \Omega$,
 
 $p(\rho)$ ~--- давление газа, эта функция считается известной. 
 
 Введём новую функцию 
\end{document}
